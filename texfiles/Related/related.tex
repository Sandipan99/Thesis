\chapter[Related Work]{Related Work} In this chapter, we discuss relevant studies related to the objectives of this thesis. Currently video related traffic is the dominating traffic and this trend is going to continue; we discuss related studies about enhancing offloading performance for Video on Demand (VoD) in section~\ref{sec:offload_related}. We have addressed the issue of uneven load distribution using association control protocol; section~\ref{sec:hetero_related} elaborates the related studies. Restriction of unauthorized traffic can be achieved by means of password sharing restriction; section~\ref{sec:unauthorized_related} provides overview of related studies on necessity of restricting password sharing practice and actions taken regarding this.

\section{Efficient Offload using WiFi Network} \label{sec:offload_related}
The Internet has witnessed  a sharp increase in video traffic in the recent years with all kinds of Internet streaming systems, 
such as VoD and Internet P2P-based streaming systems ~\cite{1c1,1c3,1c4} etc. And with the huge surge in smartphone, more and more accesses from mobile devices are directed to all kinds of Internet streaming services. Almost all popular streaming service providers, including Youtube\footnote{http://www.youtube.com/mobile/}, Netflix\footnote{https://www2.netflix.com/Mobile/} and Hulu\footnote{http://www.hulu.com/plus}, also provide streaming services to subscribed mobile users via APPs built in various mobile operating systems. This is virtually pushing the capacity of 3G to a limit. An important and not-so-new proposal to handle the situation is to device techniques to push 3G traffic to Wi-Fi network~\cite{5c,7c}. The users also can derive several benefits from it, for example, typically a smartphone accessing content through Wi-Fi would have its energy drained at a much lower rate than it would be through 3G~\cite{9c,8c}, the pricing for data download which is still being widely discussed and debated~\cite{
15c} would certainly go down. However, there are flip sides in trying to offload 
streaming services to Wi-Fi. Wi-Fi delivers high throughput when connected, but in a mobile scenario, it suffers from frequent disconnections even in a commercially operated, metro-scale deployment~\cite{10c}. Hence, the nature of offload has primarily being directed towards offloading delay tolerant services~\cite{offloading_wifi_1}.
\newline

\noindent Side by side there have been several efforts to (a). devise lightweight / efficient video streaming algorithm to make it more suitable for mobile devices~\cite{12c}, (b). improve the Wi-Fi protocol to ensure seamlessness. Due to mobility, typically duration of association with an Wi-Fi AP is very short, and hence, protocols to ensure fast association need to be established. An important attempt in this direction is the proposal of Cabernet Transport Protocol (CTP) ~\cite{13c} which incorporates fully automatic scanning, AP selection, association, DHCP negotiation, address resolution, and verification of end-to-end connectivity, as well as detection of the loss of connectivity. Authors of~\cite{8c} propose a smart, energy-efficient and fast way to detect and connect Wi-Fi.~\cite{10c} investigate a transport layer protocol design that integrates 3G and Wi-Fi networks, specifically targeting vehicular mobility. The goal is to move load from the expensive 3G network to the less expensive Wi-Fi network 
without hurting the user experience. 
\newline

\noindent There have been also important works ~\cite{5c, fast_handover_1, fast_handover_2, fast_handover_3} to ensure fast switching to 3G in order to overcome the poor availability and performance glitches of Wi-Fi. All these efforts are improving the reliability of Wi-Fi and making it suitable for streaming applications. For example, new standards 821.11e and 802.11p, 802.11-2012~\cite{26c} have been proposed which ensure QoS and mobility respectively.\\

To reduce access time, content centric network is becoming popular. In such network, routers are storing files along with routing path~\cite{Psaras,Ming,Chai,cho12}. In similar line, base-station as edge-server for caching is explored in different network like in cellular network~\cite{cell}, Wi-Fi network~\cite{same}. In light of such developments, we believe our proposal of carefully placing content directly into APs is timely and would enhance its usability in offloading streaming content.

\section{Managing Heterogeneous Traffic}\label{sec:hetero_related}
Heterogeneous traffic possess an important problem in wireless network. Heterogeneity comes due to uneven spatial distribution~\cite{uneven_spatial_load_1,uneven_spatial_load_2,Bahl_2007,uneven_load_bejerano} or due to different kinds of applications with different QoS requirements~\cite{application_heterogeneity,multiclass_traffic_1}. Association control protocol is a promising solution to handle this heterogeneity. Being an important issue several attempts have been made from both academics and industry to address this issue. The problem has been studied on different kinds of network (e.g. WLAN, cellular). Association control problem is studied to optimize network, mobility and traffic parameter such as changing load distribution, QoS, energy, throughput, interference, handling traffic from heterogeneous networks, handoff etc. In this section we discuss the related studies.\\

Mobility is an important phenomenon in wireless networks. Due to mobility, load across base stations changes continuously. Focus of many association control studies is to handle this changing load across base stations. In~\cite{dynamic_load_balancing_1}, authors propose a two-tier load balancing algorithm which assumes a centralized server has information of channel quality between all base stations and terminals. With this knowledge, scheduling scheme could avoid interference and hence performance gain is achieved. Performance gain is higher when load distribution is asymmetric. In~\cite{dynamic_load_balancing_2}, authors propose a load balancing algorithm for WLAN and cellular integrated architecture. Dynamic decision of vertical handoff and quality of service guarantee during admission is ensured through a two-phase strategy.
In~\cite{bejerano2007,bejerano2004,Sarkar_2009} authors propose methods of load balancing by simultaneous association of a device with multiple APs. In~\cite{arunesh_infocom_06,arunesh_mobicom_06} authors tried to balance load of wireless LAN using proper channel management. In another approach, authors try to balance the load by transferring the load to less loaded neighbors~\cite{anand2002hot,huazhi2008,velayos2004}.\\

In recent years, we are experiencing co-existence of a number of wireless technology in same geographical location. Many studies have been done for association control in such heterogeneous network environment. In~\cite{heterogeneous_2}, authors propose an association control algorithm for heterogeneous network consisting of 3GPP, LTE and WLAN. Areas under cellular coverage is divided into several zone, and for each zone requirement of network resources is computed. Call blocking and handoff failure probabilities based on load of different networks facilitate the admission control decision. In~\cite{heterogeneous_3}, authors propose a unified call admission scheme to support multimedia in a highly heterogeneous network. This scheme can easily adopt load variation with minimal computation. Admission control decision is taken based on Markov decision process. Interestingly, with this framework objective of admission control can be varied. In~\cite{heterogeneous_4}, authors propose joint admission control 
scheme which utilizes system resources optimally to ensure QoS of already admitted call. It also reduces probability of new call blocking and call dropping during handoff.\\

Continuity of an ongoing call is an important issue. There is a significant chance of call dropping during handoff. Hence, another set of studies are done on admission control to minimize call dropping during handoff. In~\cite{call_admission_handoff_1}, authors proposed a call admission framework which applies soft-QoS based handoff strategy. A non-linear optimization problem is formulated to minimize call dropping probability. Branch-and-Bound strategy is used to keep this optimization problem tractable. In~\cite{call_admission_handoff_2}, authors proposed a call admission scheme to facilitate vertical handoff between WLAN and UMTS network. This strategy differentiates service class and meets the QoS requirement of higher priority class while maintain minimum requirement of lower priority class. In~\cite{call_admission_handoff_3}, authors proposed an effective call admission control strategy with service differentiation for QoS provisioning and efficient resource utilization and seamless handoff. In another 
approach authors used to predict mobility and accordingly appropriate AP is chosen for handoff~\cite{Kim_infocom_08}.\\

In e-commerce, it is observed that a longer session typically ends with purchase. So, it is important to ensure connection of a longer session. To address this, authors of~\cite{session_based_admission_control_1} proposed a session based admission control strategy to improve web QoS for commercial web servers. It provides predictable platform for ensuring completion of web session. This strategy auto tunes parameters of admission control depending on the load. In~\cite{session_based_admission_control_2}, authors proposed a session based admission control for multi-tier Internet applications. Admission control algorithm designed for single web-server is not capable to handle bottleneck of multi-tier application as bottleneck of resources from different tier changes dynamically.\\

VoIP got immense popularity as it is significantly less costly. Congestion in WLAN is not gradual, it is often observed that with little extra traffic congestion, scenario changes significantly and may result in severe degradation in QoS of all existing VoIP communication. In~\cite{call_admission_voip_1}, authors propose a link adaptation scheme to restrict sudden drop of QoS of existing communication. If an existing communication requires more bandwidth and hence extra time slot then it may go through a decision function which allocates extra slot only if it safe otherwise user needs to go through link adaptation. In~\cite{call_admission_voip_2}, authors propose another call admission control for VoIP communication through centralized control. A central database is kept which stores information including cost data associated with individual paths and links across the network and dynamically updates the network condition. Admission control point at each base station consults with the central server before 
admitting a new call. \\

Another important issue with wireless network is interference. Many studies on association control is done to mitigate interference in network. Users on cell boundary suffer from poor throughput and severe inter-cell interference. Authors in~\cite{kson_wc_2009}, proposed a dynamic association control scheme to avoid interference by optimizing partial frequency reuse and load-balancing schemes. In~\cite{li_inter_2006}, authors proposed a two-tier architecture for dynamic radio resource allocation. They developed optimization function for interference avoidance and traffic/channel adaptation. They also did study to show trade off between sector interference suppression and dynamic interference avoidance. In~\cite{interference_3}, authors proposed an association control scheme for femtocell deployment scenario. They proposed a mechanism based on reinforcement learning for slot allocation to the traffic streams on different sub-carriers, which is employed by each femto access point to mitigate interference among 
femtocells and the underlaid macrocell. In~\cite{amit_IQU_mobicom_06}, authors proposed an interference mitigating association control scheme created from flash crowd.\\

As battery technology is not at par with other technologies, power remains an important issue for mobile devices. In~\cite{energy_aware_admission_control_1}, authors proposed an energy aware admission control protocol. They proposed two algorithms a){\em Victim Selection Algorithm} which minimizes energy consumption for new call or to handoff call and b) {\em Beneficiary Selection Algorithm} which conserves energy during completion of a call.  In~\cite{energy_aware_admission_control_2}, authors proposed a game theoretic approach for energy aware association.\\




\section{Restricting Unauthorized Traffic}\label{sec:unauthorized_related}

Authentication system can be conceptualized in a two dimensional space where one dimension is security and another dimension is usability/simplicity. A number of authentication systems are proposed to fill this two dimensional space. Text based password is the most popular authentication mechanism among the existing systems and believed to be secured~\cite{quest}. Many biometric based authentication systems have been proposed as this class of password is believed to be robust. Examples of few biometric authentication schemes are like face based authentication~\cite{Darwish}, fingerprint based authentication ~\cite{Darwish,khan2008}, iris based authentication~\cite{Granger}, audio based authentication~\cite{Chibelushi, mccool}, gait based authentication~\cite{GafurovHS06}. Graphical password is a usable authentication mechanism where a predefined graphical image is shown to a user. The user requires to touch predetermined areas of the image in a particular sequence~\cite{Blonder}. Another interesting approach 
of authentication mechanism is context based authentication~\cite{Denning,Nosseir}, which leverages the idea of individuality of users~\cite{Zorkadis}.\\

\noindent In recent years there has been observed a new trend of password sharing in our society. It has been reported substantially in many blogs and many social surveys. In a report published on $8$th May, $2013$, Cyberark (a private information security company) claims that ``$86$ percent of large enterprise organizations either do not know or have grossly underestimated the magnitude of their privileged account security problem, while more than half of them share privileged passwords internally''. Though it is quite believable that teens are more involved in sharing passwords, the survey found that the practice of password-sharing is a common phenomenon across a wide range of age groups. $64\%$ people of the age group $18-29$ share passwords, compared with $70\%$ people of the age group $30-49$, $66\%$ people of the age group $50-64$, and $69\%$ people of those over $65$\footnote{http://healthland.time.com/2014/02/24/the-complicated-politics-of-sharing-passwords-with-a-partner/}. A survey finds that $67\%$ of Internet users who are either married or in a committed relationship have shared a password to at least one online account with their spouse or partner\footnote{http://sanfrancisco.cbslocal.com/2014/02/13/pew-study-finds-many-couples-sharing-passwords-online-accounts}. In another survey with $60$ men and $62$ women, authors report that one third of the participants share their email password, while one fourth of the participants share their {\em Facebook} password with close friends and partner~\cite{self}.\\ 

\noindent This malpractice results in different kinds of unwanted scenarios including fraud in money transaction, stealing secret information, many service providers loss their revenue. A lot of events related to ATM money fraud have been reported worldwide\footnote{http://timesofindia.indiatimes.com/topic/ATM-money-theft, http://www.belfasttelegraph.co.uk/news/local-national/northern-ireland/pair-charged-over-atm-money-theft-29722570.html}. Edward Snowden, a former contractor with the National Security Agency (NSA), convinced up to two dozen NSA employees to part with their passwords and as a result he was able to download a large number of secret documents in part\footnote{http://www.theverge.com/2013/11/7/5079428/snowden-used-colleagues-passwords-to-access-secret-files-report-says}. Even a lot of service companies like {\em NetFlix}, {\em HBO Go} are losing their revenue due to sharing of passwords. According to Michael Pachter, a Wedbush Securities analyst in Los Angeles, ``As many as $10$ million people are watching the online video service without paying''\footnote{http://www.bloomberg.com/news/2013-04-22/netflix-seen-cracking-down-on-sharing-to-bolster-profit.html}. {\em Netflix}, {\em HBO Go} discourage subscribers to share their password \footnote{http://www.cnet.com/news/netflix-ceo-curse-you-password-sharing-literally/\\http://blogs.marketwatch.com/themargin/2013/12/31/netflix-takes-a-shot-against-shared-subscriptions-some-of-them-anyway/\\http://www.slate.com/blogs/moneybox/2014/03/12/hbo\_go\_sharing\_passwords\_is\_not\_encouraged\_by\_network.html}.\\

\noindent A number of social surveys regarding password practices reveal many reasons behind sharing passwords. Among them few significant reasons are - a) gaining trust of peer/societal pressure, b) ease of maintenance/administration, c) unawareness of implication. However, irrespective of the reason of sharing password, ultimate outcome is harmful in most of the scenarios. To restrict this practice few initiatives are taken from different organizations and also there are several social awareness campaign. Like there was a campaign titled ``Passwords are like underwear'' ran by the {\em Information Technology Central Services} at the {\em University of Michigan} a few years back. Bankers strictly recommend not to share password or pin even with family members. Under the {\em Electronic Funds Transfer Code of Conduct (2002)} (by {\em Australian Securities and Investments Commission}), the account holder is liable for losses where ``the user voluntarily discloses one or more of the codes to anyone, including 
a family member or friend''. Work has been also done to create policies to restrict password sharing in enterprise ~\cite{guide}. {\em Hospital Corporation of America} publishes their code of conduct where all health care employees are specifically instructed not to disclose their password to anyone else. Even universities encourage student for good practice of not sharing password\footnote{http://dafis.ucdavis.edu/userinfo/password/sharing.cfm}.\\




\noindent A number of contributions have been made by researchers in this direction. \textit{Physical biometric} is the most promising solution that can avoid sharing. However, it is not feasible to apply in all kinds of authentication scenario. Another interesting approach to restrict sharing is \textit{HCI-based biometric}. In~\cite{4151786}, authors reported a number of HCI-based biometric schemes. Authors categorized the existing schemes into two classes -1) input device interaction based biometric (keystroke, mouse, haptic), and 2) software interaction based biometric (email-behavior, programming style, computer game strategy). In input device interaction based biometric, keystroke got most attention. Authors of~\cite{keystroke1} distinguish authentic user from non-authentic users by key stroke pattern during authentication. If two users can be distinguished from key-stroke pattern then that would definitely restrict password sharing. However, ~\cite{keystroke2} authors reported even $50\%$ false 
acceptance rate in key stroke pattern based authentication. In~\cite{deVel}, authors proposed techniques to identify sender of an e-mail by mining e-mail content. One Time Password (OTP) is another potential option for solving the issue of sharing. As it changes dynamically, it would be difficult to share every time. OTP still can be shared to reduce subscription charge if user adopts the painstaking path of repeated sharing. Instead of a random string, if OTP contains some private information of user then it would become difficult to be shared by user. A lot of secret information that we know may be used for identification/authentication. For example, to identify fraud in online credit card transaction, a user might have to face different security questions like geo-location, email-address, shipping-address, previous transaction made from this credit card~\cite{Akhilomen}. Daily activities can be a reach source for extracting more and more {\em secrets}. In~\cite{sdas}, authors propose to capture daily 
events that user can remember for authentication. They fine tuned the system with two initial mechanical turk experiments. However, authors concentrate more on answer pattern of authentic users rather than designing the challenges for high recall rate for authentic users and low guessability for impostor. In~\cite{rajeshkb}, authors propose to find memorable fingerprint of cellphone data. Authors did a two stage experiments to fine tune the system and achieved $\approx$ $85\%$ successful authentication of legitimate user and $\approx$ $15\%$ successful authentication of impostors. However, in this study also, authors did not differentiate the potential of different activities and treated them equally. Moreover, they have generated questions from activities collected over a month and hence it suffers from staleness. 
