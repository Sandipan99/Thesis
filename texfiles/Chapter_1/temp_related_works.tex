\noindent
\section{Related works}
\label{relatedworks}
Most of the initial works attempted to study temporal networks by aggregating the network across all times and then analyzing this aggregated network.
However, it was found that
time ordering is an important issue and destroying this ordering information severely affects the understanding of the true nature of the network. 
Several ways have been therefore devised to represent temporal networks. Basu et al. 
have shown in ~\cite{journals/corr/abs-1012-0260} a way of representing temporal networks as a time series of static graph snapshots and have proposed a stochastic model for generating 
temporal networks. Perra et al. in ~\cite{perra2012activity} have provided an activity driven modeling of temporal networks. They define activity potential which is a time 
invariant function characterizing the agents' interactions and propose a formal model of temporal networks based on this idea. Random walks have been introduced in 
temporal networks ~\cite{starnini2012random} to single out the role of the different properties of the empirical networks. It has also been shown that random walk 
exploration is slower on temporal networks than it is on the aggregate projected network. Several other modeling frameworks for temporal networks have also 
been proposed (see ~\cite{vespignani2011modelling,hill2010dynamic,hanneke2010discrete} for references). Apart from the attempts to generate temporal networks several other properties of these networks have also been investigated.
Temporal networks of human communication has been observed to be bursty in nature in ~\cite{barabasi2005origin}. Dynamics of human face-to-face interactions have been studied in ~\cite{zhao2011social}.
~\cite{tang2009temporal} shows how the presence of burstiness affects the dynamics of diffusion process. Further different metrics to study the properties of temporal networks 
have also been proposed in ~\cite{pan2011path}.  

On the other hand, time series have found a lot of applications in economic 
analysis and financial forecasting ~\cite{hamilton1989new}. It has also been applied in tweet analysis ~\cite{o2010tweets}. 
However, temporal networks have not 
been studied in details so far as a time series problem except for preliminary attempts in ~\cite{scherrer2008description,hempel2011inner}.
\textcolor{blue}{In ~\cite{scherrer2008description} the authors analyze temporal networks as time series but to the best of our knowledge this is the first 
work which leverages the time series forecasting tools to predict the network properties at a future time instant. 
Frequency domain analysis on the time series representing 
temporal network and its implication also remain unexplored to the best of our knowledge.}
 Therefore we propose to leverage in this paper the standard 
techniques of time series analysis (both in time and frequency domain) to understand the dynamical properties of temporal networks.
Our prime contribution is to develop a unified framework to analyze and predict different properties of temporal networks based on time-series modeling.

\medskip
