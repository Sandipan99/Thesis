\noindent
\section{Description of the dataset}
\label{dataset}
We perform our experiments on five human face-to-face network datasets: INFOCOM 2006 dataset~\cite{cambridge-haggle-imote-infocom2006-2009-05-29}, SIGCOMM 2009 dataset~\cite{thlab-sigcomm2009-mobiclique-proximity-2012-07-15}\footnote{http://crawdad.org/}, 
High school datasets (2011, 2012)~\cite{fournet2014contact} and Hospital dataset~\cite{vanhems2013estimating}\footnote{http://www.sociopatterns.org/}.


% \textcolor{blue}{The first two represent human face-to-face interaction while the last two represent 
% human interaction through social media.}
\begin{itemize}
\item {\bf INFOCOM 2006:}
This is a human face-to-face communication network and was collected at the IEEE INFOCOM 2006 conference at Barcelona.
78 researchers and students participated in the experiment. They were equipped with imotes and apart from them 20 stationary imotes were deployed as 
location anchors. The stationary imotes had more powerful battery and had a radio range of about 100 meters. The dynamic imotes had a radio range of 30 meters.
If two imotes came in each others' range and stayed for at least 20 seconds then an edge was recorded between the two imotes.
The edges were recorded at every 20 seconds. 
Therefore, this is the lowest resolution at which the experiments can be potentially conducted. However, we observe that at this resolution the network is extremely sparse 
which makes it difficult to conduct meaningful data comparison and prediction. We observe experimentally that the lowest interval that allows for appropriate comparison 
and prediction is $5$ minutes and therefore we set this value as our resolution for all further analysis.
%The network is both unweighted and undirected.
 \item {\bf SIGCOMM 2009:}
 This is also a human face-to-face communication network and was collected at the SIGCOMM 2009 conference at Barcelona, Spain. The dataset contains data collected by an opportunistic mobile social 
application, MobiClique. The application was used by 76 persons during SIGCOMM 
2009 conference in Barcelona, Spain. The trace records all the nearby Bluetooth devices reported by the periodic 
Bluetooth device discoveries.
Each device performed a periodic Bluetooth device discovery every 120+/-10.24 seconds for nearby Bluetooth devices. A link was added with a device 
on discovering it. 
We remove the contacts with external Bluetooth 
devices and a network snapshot is an aggregate of data obtained for 5 minutes. 
%The network is both unweighted and undirected.

\item {\bf High school datasets:}
These are two datasets containing the temporal network of contacts between students in a high school in Marseilles taken during December 2011 and November 2012 respectively. 
Contacts were recorded at intervals of 20 seconds. We consider a network snapshot as an aggregate of data obtained for 5 minutes. 
%The network is both unweighted and undirected.

\item {\bf Hospital dataset:}
This dataset consists of the temporal network of contacts between patients and health care workers in a hospital ward in Lyon, france. Data was collected at every 20 second intervals.
 Due to sparseness of the network of 20 seconds, we consider each network snapshot as an aggregated network of 5 minutes. 
 
 
 In table 1 we provide the details of the datasets.  
%  \item {\bf Twitter Hashtag co-occurrence network:}
%  This is a hashtag co-occurrence network created from the tweets posted by human users 
%  on London Olympics from 27th July 2012 to 1st August 2012 extracted from 1\% random sample.
%  The network snapshot is an aggregated network of 10 minutes each. Each node represents a unique hashtag and a link appears between 
%  two nodes if they co-occur in a tweet. The network consists of 7118 nodes i.e., unique hashtags. It is both unweighted and undirected.
%  \item{\bf Facebook posts network:}
%  \textcolor{blue}{The original network consists of a set of Facebook users (nodes) and a link appears between two users if one posts on other's wall. As it is a directed network, we need to first obtain an undirected projection of it.   
%  We divide the whole data into slots of 6 hours and a link appears in the undirected projection between two nodes if both the users have posted in each other's wall within this time window. 
%  Thus we obtain an undirected projection of the original directed network and we perform our analysis on this network. Note that due to this conversion we found several 
%  such 6 hour slots without any edge making the series discontinuous.
%  All our further analysis is on the part of the whole window where we have continuous datapoints for the longest stretch.
%  %We take out only that part of the whole time window with a large sequence of consecutive 
%  %slots. Thus the final network stretches for this time window only.
%  The network consists of 20284 unique users.}
%  
 \end{itemize}
 
\begin{table*}
\centering
\begin{adjustbox}{max width=\textwidth}
\begin{tabular}{|c|c|c|c|c|c|}
\hline
Dataset         & \# unique nodes & \# unique edges & edge type  & \begin{tabular}[c]{@{}l@{}}Time span of \\ the dataset\end{tabular}     & \begin{tabular}[c]{@{}l@{}}Time steps \\ for  prediction\end{tabular} \\ \hline
INFOCOM 2006    & 98              & 4414            & undirected &  1120                                                                   & 200 - 800                                                             \\ \hline
SIGCOMM 2009    & 76              & 2082            & do         &  1068                                                                   & 300 - 900                                                             \\ \hline
Highschool 2011 & 126             & 5758            & do         &  1215                                                                   & 200 - 900                                                           \\ \hline
Highschool 2012 & 180             & 8384            & do         &  1512                                                                   & 200 - 1000                                                            \\ \hline
Hospital        & 75              & 5704            & do         &  1158                                                                   & 100 - 900                                                             \\ \hline
\end{tabular}
\end{adjustbox}
\label{tab_data}
\caption{Properties of the dataset used.}
\end{table*}
\medskip
