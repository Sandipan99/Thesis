\noindent
\section{Introduction}
\label{introduction}
\if{0}
Recently the research community is reaching a consensus that many real 
networks have nodes and edges entering or leaving the system dynamically, thus making temporal networks a better way of representing these networks. 
Initial studies on temporal networks have been performed by aggregating the nodes and edges over all time 
steps and then analyzing the behavior of the aggregated network. This strategy however hides the time ordering of the nodes and the edges 
which may have a significant role in the understanding of 
the true nature of such temporal networks.
Researchers have subsequently come up with growth models and have proposed several metrics. Tang et al. have showed in ~\cite{TSMML10:smallworld} 
the presence of correlation and small world 
behavior in temporal networks. In ~\cite{stehle2010dynamical} the temporal network of human communication has been identified to be bursty in nature.
\fi
Understanding the structural properties of temporal network is of primary importance. To this end the research community was initially inclined towards 
aggregating the nodes and edges over all time 
steps and then analyzing the behavior of the aggregated network. This strategy however was found to hide the time ordering of the nodes and the edges 
which plays a significant role in understanding  
the true nature of such temporal networks. This led researchers to come up with various growth 
% To the best of our knowledge no work has been done which provides a time series analysis of temporal networks in both 
% time and frequency domains. 
%In ~\cite{shimada2012networks} Shimada et al. propose a framework for transforming a static complex network to a time series using classical multi-dimensional scaling.
Recently, new network applications have cropped up where an estimate of the network properties are helpful even though the network structure is itself unavailable. 
For instance, in order to launch a targeted attack on the network one might not require the full knowledge of the network structure. Instead, an approximate estimate of some 
of the properties might be useful in finding the order in which the nodes and the edges may be removed. 

In this work, 
we propose a simple strategy to represent a temporal network as time series. Essentially, we consider a temporal network as a set of static snapshots collected 
at consecutive time intervals and represent each of them in terms of the properties of the network. In specific, we consider eight properties namely number of 
active nodes, average degree, clustering coefficient, number of active edges, betweenness centrality, closeness 
centrality, modularity and edge-emergence ~\cite{sur2014attack}. 
%Note that this framework serves as a remarkable handle for transforming a network analysis problem to a classical time series analysis problem. Indeed

We then use the known analytical tools for time series predictions to predict the network properties at a future time instance. 
Note that the time series framework can be particularly effective as it is impossible to 
define a unified network evolution/growth model for temporal networks simply because the  
rules of temporality are varied across systems. Hence the feasible alternative could be to learn 
the evolution pattern (which we do through time-series analysis) and then predict the later time steps. 

Due to various irregularities in the time series, predictions at certain points are erroneous. Therefore we 
further refine our prediction framework using spectrogram analysis by identifying beforehand the cases where the 
prediction error is high i.e., unsuitable for prediction. In fact we observe that the accuracy of the framework is 
enhanced further by 7.96 (for error level $\leq 20\%$) on an average across all datasets 
if we remove 
the cases which are deemed unsuitable for prediction by spectrogram analysis.

As an application we also propose a strategy to launch targeted attacks based on our prediction framework and show that this scheme beats the state-of-the-art ranking method 
used for such attacks. We believe that our framework could be used in designing ranking schemes for nodes in temporal networks at a future time step  
albeit the network structure at that time step itself is unknown.
%\textcolor{blue} {%In particular, 
% We perform our experiments on two types of human face-to-face communication network, 
% we observe that the properties could be segregated based on time domain and frequency domain analysis 

%  As an additional objective, we perform frequency domain 
%  analysis of these time series in order to enrich our predictions. 
%We use three types of data sources for our experiments: (i) time-varying human face-to-face interaction networks, (ii) Twitter hashtag co-occurrence graph 
%varying over the period of an event (e,g., London Olympics) and (iii) Facebook posts graph of a set of users. 

%\todo{This paragraph needs to be discussed}
%We report our prediction results on real world time-varying networks. 
We perform our experiments on five different human face-to-face communication networks 
and observe that the above properties could be segregated based on time domain and frequency domain (spectrogram) characteristics. 
In general this method allows us to make predictions with very low errors.   
 Importantly, the frequency domain analysis also nicely separates out those properties that can be predicted with low errors from those for which it is not 
 possible.
\if{0} 
Rest of the paper is organized as follows. In Section ~\ref{relatedworks} we present a brief review of the literature. 
Section ~\ref{mapping} describes the framework for mapping temporal networks to time series. 
Section ~\ref{dataset} provides a brief description of the datasets we use for 
our experiments.
  In section ~\ref{properties} we perform a detailed time domain and frequency domain analysis of the time series. 
Section ~\ref{prediction} outlines the description of our prediction framework. In Section ~\ref{result} we provide the detailed results of our prediction framework on the 
human face-to-face communication networks. We also show how the prediction scheme could be enhanced using the spectrogram analysis. We further 
propose an attack strategy based on the prediction scheme and show that it beats the state-of-the-art methods (section ~\ref{attack}). 
We conclude in section ~\ref{conclusion} by summarizing our main contributions and pointing to certain future directions.
\fi
\medskip
