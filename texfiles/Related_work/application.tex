\section{Survey on Post-hoc Analysis of Communities}
In this section, we survey the current literature pertaining to the analysis of detected communities and how this community information can
be used in the development of various systems. 

\subsection{Tracking Communities over Time}
In real world, the membership of communities tend to change gradually. Backstrom et al. observed this on the communities of LiveJournal
users
and communities of conference publications on DBLP \cite{Backstrom:2006}. So it is important not only to detect communities but also to
track the changes in membership over time. The questions are: which community in one snapshot metamorphoses into anther in the next
snapshot? How and what fraction of membership changes in between? The problem of tracking communities is
motivated by a problem in behavioral ecology in studying animals that live in fission-fusion societies such as zebra and the Asiatic wild
ass \cite{citeulike:938243}. A natural question is
which group that we observe today is the same as that was previously observed. Groups in this setting are manifestations of perpetual
communities. Inversely, a community is a consistent string of groups seen on different days. This is the problem of tracking communities
over time. Loosely speaking, it is about how to string different groups from the same day into communities which span over multiple days. 

There is a handful of work specifically on the problem of tracking communities over time. Berger-Wolf and Saia \cite{Berger-Wolf:2006}
proposed a framework which defines communities as independent local patterns. There, a community (or, metagroup) is a sequence of groups
which have sufficiently high similarity. The similarity between two groups is the number of common members normalized by the sizes of the
two groups. Characteristics of communities are studied via community-based statistical measures such as number of all possible communities,
their sizes and life spans. Also, they proposed an approach to study the survival of the communities via finding a critical set of groups
whose removal leaves only short-lived communities.

Spiliopoulou et al. \cite{Spiliopoulou:2006} proposed a framework, called {\em MONIC}, for tracking communities over time. The framework
utilizes a similarity function of groups at different time steps. The function takes into account the number of common members, the sizes of
the groups, and the time decay between the groups. Then, two groups are strung together as being in the same community if their similarity
is above a certain threshold. The framework not only strings groups into communities but also detects splitting and merging of communities
by a separate set of threshold parameters. 

Tantipathananandh et al. \cite{Tantipathananandh:2007} proposed the first framework which rigorously formulates the problem of tracking
communities as an optimization
problem. Although the appealing aspect of this framework is the social costs model which
has its roots in the
social sciences view of group dynamics \cite{Pearson_drif}, the framework has a strong assumption that all time steps must have the same
length. Tantipathananandh et al. \cite{Tantipathananandh:2009} further introduced an improved framework which can handle data with time
steps
of variable length. 

\subsection{Analyzing Community Evolution in Networks}
Slightly different from the task of community tracking is the study of the evolution of communities over time. This problem
attracts a lot of
research interest due to its enormous applications in real-world scenario. For example, in a blog network we might wish to detect which
communities of blogs are relatively stable in size over a period of time \cite{Lin:2008}. In a mobile phone network, changes in community
size over a timeframe can reveal calling patterns and customer churns \cite{5562773,citeulike:1206611}. In other contexts such as scientific
collaboration networks, communities of researchers that span many years suggest long-term research collaboration \cite{citeulike:1206611}.
Such communities can be further investigated to identify researchers in particular fields who are consistently productive over a period of
time \cite{Lin:2008}. Previous work along this line analyzed community changes using a life-cycle model comprising events such as birth,
death, expand, contract, merge, and split \cite{5562773,citeulike:1206611}. Asur et al. \cite{Asur:2009} emphasized the life-cycle of nodes,
an emphasis that is impractical in networks with millions of nodes and
irrelevant when an overview of how communities evolve is required. Palla et al. \cite{citeulike:1206611} used the above events to quantify
the evolution of a phone call network and a coauthorship network, whereas Greene et al. \cite{5562773} used the events to investigate
community evolution in a phone call network. Except for \cite{citeulike:1206611}, little attention has been paid to modeling an event as a
function of time. Recently,  Lu\"{z}ar et al. \cite{Povh} studied interdisciplinarity of research communities detected in the coauthorship
network of Slovenian scientists over time.

\subsection{Community Structure in Link Prediction}
The information of community of nodes can also be leveraged in the task of link prediction. Clauset et al. \cite{Clauset.Moore} proposed a
method to determine the hierarchical structure of a network by using MCMC sampling to create a binary dendrogram that joins nodes into
groups. Since this method got introduced, a variety of similar methods and models have been proposed. Valverde-Rebaza and de Andrade
Lopes \cite{6412391} described experiments to analyze the viability of applying the within and inter cluster (WIC) measure for predicting
the existence of a future link on a large-scale online social network. They further proposed three measures for the link prediction task
which take into account all different communities that users belong to \cite{Valverde-RebazaICSA2014}. Sachan and Ichise \cite{5460690}
proposed to build a link predictor in a co-authorship network, and showed that the knowledge of a pair of researchers lying in the same
dense community can be used to improve the accuracy of our predictor further. Recently, Fenhua et al. \cite{Li2014432} proposed a link
prediction method based on clustering and global information.

\subsection{Community Structure in Information Diffusion}
Communities are vehicles for efficiently disseminating news, rumors, and opinions in human social networks. Several approaches studied this
phenomenon using the community structure of the network. Belak et al. \cite{6425767} studied information diffusion across communities and
showed that one can achieve high community-based spreading using an efficient targeting strategy. Nematzadeh et al. \cite{Nematzadeh}
used
the linear threshold model to systematically study how community structure affects global information diffusion. Kimura et al.
\cite{KimuraYSM08} used community analysis to find  influential nodes for information diffusion on a social network under the independent
cascade model. Weng et al. \cite{Weng13} focused on understanding interactions between community structures and information diffusion, and
developed predictive models of information diffusion based on community structure. Chen et al. \cite{Chen20121848} employed the network to
investigate the impact of overlapping community structure on susceptible-infected-susceptible (SIS) epidemic spreading process. Similarly,
Xiangwei et al. \cite{Xiangwei}  studied epidemic spreading in weighted scale-free networks with community structure. Recently, Shang et
al. \cite{Shang6903136} classified vertices into overlapping and non-overlapping ones, and investigated in detail how they affect epidemic
spreading.

\subsection{Community Structure in Recommendation Systems}
Community detection algorithms and clustering functions constitute a powerful tool in the development of network based recommendation
system.
Zhuhadar et al. \cite{Zhuhadar} used the community detection method to design a visual recommender system to recommend learning resources to
cyberlearners within the same community. Lisboa et al. \cite{Isboa} proposed a method to improve recommendation systems by taking into
consideration changes in the behavior of users over time. For that, communities are first detected using a network analysis method and
recommendations are made for each community using Na\"{i}ve Bayes modeling. Kamahara et al. \cite{Kamahara:2005} proposed a recommendation
method in which a user can find new interests that are partially similar to the user's taste, where partial similarity is an aspect of the
user's preference which is projected by the community in which the user belongs. Musto et al. \cite{Musto_atag} particularly studied user
community behavior in OSN and developed {\em STaR} to suggest a set of relevant keywords for the resources to be annotated. Fatemi and
Tokarchuk \cite{Fatemi} proposed novel community based social recommender system,  {\em CBSRS} which utilizes the social data to provide
personalized recommendations based on communities constructed from the users' social interaction history with the items in the target
domain.