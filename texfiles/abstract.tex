

\thispagestyle{empty}
Real-world systems are mostly dynamic in nature and hence can be accurately represented as time-varying networks. Understanding different characteristics 
of these time-varying networks is of prime importance as it is essential in accurately discerning the true behavior of the underlying real-world systems. In this thesis we look into different aspects of time-varying networks: (i) predicting properties of temporal network at future time steps, (ii) designing sampling algorithms for streaming
graphs such that the underlying community structure in the original graph is preserved, (iii) designing information diffusion model and provide theoretical framework
that captures the uneven susceptibility of agents in a system and (iv) leveraging network based and other data mining techniques to improve the scientific peer-review system.
While the first two objectives are related to the structural aspect of temporal networks, the third one pertains to its function and the last one demonstrates a real-world application.\\
{\bf Structure}: We propose a generic framework, whereby, we map temporal network to time series and leverage time series forecasting tools to predict the properties at a future
time step. In fact, we are able to further refine our predictions through spectrogram analysis.\\
Many real-world networks can be represented as a stream of edges arriving sequentially over time and the sheer volumes of edges as well as the dynamic nature makes it impossible to retain all the edges. Hence, one resorts to sampling a set of edges with the aim of preserving the properties of the original graph in the sample. Here we formulate a sampling algorithm ~\compas~ which is able to retain the underlying community structure of the original graph in the sample.\\
{\bf Function}: We next propose an information diffusion model inspired by the fact that every agent within a population is not equally susceptible to a disease
or equally amenable to a rumor - the ones which have been exposed more number of times (in the recent past) are more amenable. In fact, we observe that such a diffusion process progresses in two distinct phases - (i) a slow initial phase followed by a (ii) residual phase where the rate of diffusion increases manifold. \\
{\bf Application}: Finally, as an application of time-varying network analysis we consider the scientific peer-review system which presents an unique case of temporal system as the interactions between its entities could be modeled as time-varying networks. More specifically, we leverage temporal network analysis and other data mining techniques to improve the scientific peer-review system.
 


\medskip


\noindent \textbf{Keywords:} time-varying network, time series, sampling, community, information diffusion, scientific peer-review system 
