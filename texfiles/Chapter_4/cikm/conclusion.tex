\noindent

In this paper we provided a framework for identifying anomalous reviewers and editors based on  their review history considering Journal of High Energy Physics as a case study. We identified several factors that are indicative of anomalous behavior of the editors as well as reviewers. In specific for editors we observed that - 
(i) high frequency of assignment, (ii) selecting from a very small set of referees for reviewing, (iii) assigning same reviewer to papers of same author and (iv) assigning herself as reviewer instead of assigning someone else could be indicative of anomalous behavior of the editor. 

Similarly for reviewers we observe that - 
(i) high frequency of assignment, (ii) delay in sending report, (iii) assignment from only a single editor or a very small set of editors (iv) 
very high or very low acceptance ratio and (vi) delay in notifying the editor about her decision to decline are also indicative of anomalous behavior and often leads to under-performance. Based on these factors we were able to identify anomalous referees and editors using an unsupervised clustering approach. 

\noindent{\bf Future directions:} We believe our findings could be useful in better assignment of editors and reviewers and thereby improve the performance of the peer-review system. Assigning good reviewers is an important part of the peer-review process and our findings allow for identifying under-performing referees. This could be a first step towards developing a reviewer-recommendation system whereby the editors are recommended a set of reviewers based on their performance. 
We plan to come up with such a system in subsequent works.
