\noindent
\section{Conclusion}
\label{conclusion}
We in this paper performed a detailed comparative analysis of cases where a single reviewer was involved in the peer-review process and where multiple reviewers were involved, considering the review history of papers submitted to two leading journals of physics (JHEP and JSTAT). 
%Although multiple reviewers are generally preferred over a single reviewer, 
We made an interesting observation  that accepted papers which were reviewed by a single referee on average tend to garner a larger number of citations in 
long term compared to those which were reviewed by multiple referees. An exact opposite behavior is observed in case of rejected papers. 

Through a more detailed analysis, we however observed several contradictory trends - although on average single reviewers perform better, however, 
most impactful papers are largely multi-refereed;  the multi-refereed papers generally perform poorly due to discordance between the reviewers.
Further we tried to understand the reason behind under-performance and found that frequent assigning of review to a reviewer leads to his performance deterioration. 
Also those reviewers who have a tendency to be too critical or too liberal fail to identify the real good papers. 

A real problem in the process of reviewing is the scarcity of reviewers, hence discarding under-performing reviewers may not be a practical solution. 
So we checked the performance of these reviewers with high-graded reviewers and find that the quality is much better than their normal average. Hence a 
technique would be to combine high and low graded reviewers intelligently. 

\if{0}
 of review reports of referees in a multi-review cases we observed that the referees often contradict each other. On analyzing the behavior of the editors and the referees, we further observed that - \\
(i) Anomalous editors on average tend to assign multiple reviewers compared to the normal ones. Furthermore they tend to assign at least one anomalous reviewer among the multiple reviewers. \\
(ii) Anomalous reviewers when part of a multi-reviewer set up tend to perform better compared to being assigned as a single reviewer.\\

We further proposed to quantify the fitness of the referees based on the time till his last assignment and his accept ratio. 
\fi
Based on the observations above, we proposed a scheme based on genetic algorithm to recommend reviewer groups to the editors 
to make reviewer assignments. In fact our system was correctly able to recommend reviewer groups in $\sim$ 78\% cases across the two datasets.
%$78.25\%$ of cases for JHEP and $76.5\%$ of cases for JSTAT.

\if{0}
We further showed that the importance of editor's intervention to the effectiveness of our  proposed framework in long term. 
\subsubsection{Future directions}
Our current work points out to several future directions which we plan to explore in subsequent works. In its current state our algorithm only recommends referees from an already existing set. An immediate extension is to identify mechanisms to handle new referees in the system. In future, we would also like to test our recommendation system in a real settings and observe if indeed the performance of peer-review is enhanced. We believe our findings would make the scientific peer-review system much more effective.
\fi

\medskip
