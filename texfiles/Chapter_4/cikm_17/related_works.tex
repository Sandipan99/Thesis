\noindent
\section{Related work}
\label{related_works}
The effectiveness of peer-review system has often been debated. Researchers have pointed out its several limitations~\cite{ingelfinger1974peer,relman1989good,smith2006peer}. 
In fact, authors in~\cite{cole1981chance} showed that 
reviewers often fail to reach consensus when judging the quality of the contribution while~\cite{braatz2014papers} points out that 
rejected papers are often cited more in the long run. Nevertheless, little work has been done toward systematic analysis and improvement of the 
system~\cite{graffy2006improving}. That blinding can improve the quality of the reviews, has been demonstrated in~\cite{mcnutt1990effects}.~\cite{caswellimproving} pointed 
out that incentives for reviewers could encourage them doing a better job. In~\cite{sikdar2016anomalies} the authors 
point out several indicators for anomalous behavior of the referees and the editors. In fact, they were able to filter out 
anomalous editors and reviewers leveraging anomaly detection algorithms. 

On the other hand, there have been a lot of work on the formation of teams of experts whose goal is to complete a given 
project~\cite{anagnostopoulos2010power,anagnostopoulos2012online, lappas2009finding,agrawal2014grouping,pragarauskas2012multi}. A trend in this line of work is to formulate
the team formation problem as an integer linear program
(ILP), and then focus on finding an optimal match
between people and the demanded functional requirements.
Widely used techniques in solving these problems include simulated
annealing~\cite{baykasoglu2007project}, branch-and-cut~\cite{zzkarian1999forming} or genetic algorithms~\cite{ani2010method}.~\cite{agrawal2014grouping} proposed 
a way of creating study groups in an educational setting so that the overall gain for the 
students is maximized. Another set of works leverage the underlying social network among the individuals as a proxy for compatibility and propose 
techniques for creating groups so as to match the requirements 
of a co-operative task~\cite{majumder2012capacitated,mcdonald2003recommending,wolf2009mining,li2010team}. 
In this paper we consider the review information of two leading scientific 
journals (JHEP -- Journal of High Energy Physics and JSTAT -- Journal of Statistical Mechanics: Theory and Experiment) and propose a scheme 
for allocating referees for a submitted paper. Anonymity of the reviewers and absence of any kind of underlying network 
prevents us from using any of the existing network based approaches. Formulating it as an ILP also appears difficult due to unavailability of any obvious optimization function.
To the best of our knowledge this is the first work which formulates the referee recommendation in peer-review system as a group formation problem 
leveraging genetic algorithms. We believe our findings could be useful in increasing the effectiveness of the peer-review system.

\begin{table}[]
\centering
\caption{Some general information related to the two datasets.}
\label{tab:data}
\scalebox{0.85}{
\begin{tabular}{l|l|l}
\hline
                                                                                      & JHEP  & JSTAT \\ \hline\hline
\# papers                                                                             & 28871 & 6106  \\ \hline
\# accepted papers                                                                    & 20384 & 3528  \\ \hline
\begin{tabular}[c]{@{}l@{}}Fraction of multi-reviewed \\ papers\end{tabular}           & 0.12  & 0.43  \\ \hline
\begin{tabular}[c]{@{}l@{}}\# Editors with at least one \\ assignment\end{tabular}    & 95    & 148   \\ \hline
\begin{tabular}[c]{@{}l@{}}\# Reviewers with at least \\ one assignment\end{tabular}  & 3976  & 2647     \\ \hline
\begin{tabular}[c]{@{}l@{}}Average number of \\ reviewers per paper\end{tabular}      &  1.03     & 1.42      \\ \hline
\begin{tabular}[c]{@{}l@{}}Average number of \\ authors per paper\end{tabular}        &  2.87     & 2.32      \\ \hline
\begin{tabular}[c]{@{}l@{}}Average number of \\ assignments per reviewer\end{tabular} &  6.48     &  2.56     \\ \hline
\# Unique keywords                                                                    & 201 & 562  \\ \hline
\end{tabular}}
\end{table}

\medskip
