%\noindent

In this paper, we provided a framework for predicting the long-term citation of the paper which can be extremely helpful in assisting the editors in deciding whether to accept, reject or opt for a third opinion. We demonstrated that very simple positional properties extracted from the reviewer-reviewer interaction network are exceedingly important in determining the long-term citation of the paper. In specific, if we plug in these features in a regression model, we obtain $R^2$ = \textbf{0.79} and $RMSE$ = \textbf{0.496} in predicting the long-term citation of a paper. In addition, we also introduce a set of supporting features, based on the various properties of the paper, the authors and the assigned referees which further improved the prediction ($R^2$ = \textbf{0.81} and $RMSE$ = \textbf{0.46}). 

In the process of designing these features, we also made some key observations which are summarized below - \\ 
(i) the papers which went through lesser number of review rounds tend to be cited more on an average while the papers that were accepted after going through higher number of rounds are cited less on an average (although exceptions exist for both cases); 
(ii) although the reviewers tend to avoid highly polar words (negative or positive) in their review reports, the overall sentiment in the reports of accepted papers is more positive whereas the same is more negative for the rejected papers; 
(iii) the authors with higher acceptance ratio tend to be cited more on an average compared to those with lower acceptance ratio 
(iv) reviewers excessively accepting or rejecting most of the assigned papers often fail to correctly judge the quality of the paper;
(v) deeper investigation of the irregular cases revealed that the reputation of the author is often influential in the acceptance or rejection of the paper;
Apart from being a large-scale study that attempts to provide quantitative evidences supporting its necessity, ours is the first work that proposes definitive ways of improving the effectiveness of the scientific peer-review system.  

\medskip 