%\vspace{-3mm}

\section{Introduction}

%\begin{enumerate}
%\item Although there are many existing graph sampling algorithms not all properties are represented correctly in the obtained sample. For example random not sampling is able to mimic the degree distribution of the original graph but fails to retain the clustering coefficient. We hence propose that separate sampling strategies should be designed in order to retain a specific property in the sample. We consider community structure i.e., we develop a sampling algorithm which retains the community structure of the original graph. The motivation behind such a algorithm is that generally community detection algorithm are generally computationally expensive and a sample which retains the community structure of the original graph would allow for running community detection algorithms on it and then extend it to the large graph thereby saving computing resources. It would be particularly helpful in cases where it is not known beforehand which community detection algorithm to use and hence all of them need to be executed in order to obtain the best result. One can run all the algorithms on the sample and decide on using a particular algorithm.

One of the fundamental techniques to analyze very large-scale graphs is through sampling~\cite{leskovec2006sampling}, especially where the analysis on the entire graph is intractable (and often impractical).
A good sampling method should usually target a specific application and essentially preserve a set of (not all) properties of the original graph geared toward the application. For instance, a sampling method
designed for information diffusion should preserve the hubs (high-degree nodes) in the sample; whereas, a sampling scheme for outbreak detection (such as disease outbreak) should preserve the nodes with high local clustering coefficient.
Sampling has been studied extensively in the context of {\em static graphs}~\cite{leskovec2006sampling,gjoka2010walking,maiya2010sampling,rasti2009respondent,ribeiro2010estimating}; 
however, there has been very limited work on sampling from {\em streaming graphs}~\cite{henzinger1998computing} where nodes/edges arrive in discrete time intervals and only a part of the entire graph is available for analysis at any point of time~\cite{aggarwal2011outlier,ahmed2014network,lim2015mascot,de2016tri}. 

Existing graph sampling methods are mostly designed for preserving simple structural properties 
(such as degree distribution, clustering coefficient etc.) of the original graph in the sample - only few works attempted
 to
preserve complex properties like community~\cite{tong2016novel,maiya2010sampling} - which may be useful for designing a wide range of applications.
For instance, in marketing, surveys often seek to construct 
samples from different communities to capture  the diversity of the population (also known as \textit{cluster sampling})~\cite{kolascyk2013statistical}. 
In this work, we propose a novel sampling algorithm that preserves the original {\em community structure}\footnote{Here we consider disjoint community structure.} of streaming graphs. Our work sharply contrasts the recently proposed Green Algorithm (GA)~\cite{tong2016novel} which, is explicitly designed to generate a sample that preserves the community structure for {\em static graphs}. 
%A sample consisting of members from diverse communities has several important applications and hence the ability to construct one is of immense importance. 
%Many popular community detection algorithms considered to be accurate are also computationally expensive~\cite{danon2005comparing,fortunato2010community}. Representative graph sampling, then, provides a potential solution for inferring and approximating global, latent properties such as these in large graphs. By sampling a representative subgraph, analysis can be performed on the sample instead of the larger graph. Results could, then, be generalized
%to the larger population, which, in this case, is the original graph. 
%However, if one still intends to run an accurate and computationally-expensive community detection algorithm on large graphs
%and is confused which one algorithm to choose, she can run several potential algorithms on the sample graph and choose one that performs best. Then that algorithm can be used to detect communities from the original large graph.

\noindent{\bf Our contributions:} In this work, we propose~\compas, a novel sampling algorithm on streaming graph (most realistic graph representation~\cite{aggarwal2011outlier,ahmed2014network}) that is capable of
 preserving 
 %\noteng{preserving??} 
 the community structure of the original graph. 
 %while (\textit{minimally})  affecting the other standard graph properties. 
% \noteng{we are not showing this?} %\color{green}
\compas~is designed based on a novel hypothesis that  graph sampling and community detection can be \textit{interwoven} together %so that one gets benefits from the other}
 to produce a more representative sample. In particular, our contributions are the following: 
\begin{itemize}
\item To the best of our knowledge \compas~is the first {\em community-preserving} sampling method for {\em streaming graphs}. Along with the sample nodes, \compas~also outputs the community structure of the sample that closely corresponds to the community structure of the original graph.

\item In the absence of any other community preserving sampling algorithm for streaming graphs, we resort to comparing \compas~ with GA~\cite{tong2016novel} which was designed to preserve the community structure while sampling from \textit{static} graphs. Note that GA, unlike \compas, has the information of the full graph while sampling and building the community structure. Empirical evidences on synthetic and real-world graphs demonstrate that the sample generated by \compas~correctly preserves the community structure 
%but also quite faithfully reproduces other general graph properties. %\TODO{Report numbers here. For instance, our results are x\% better in preserving the community structure compared to the most competitive baseline. Similarly report for other properties.}
with average performance reaching as high as 73.2\% of GA. Further, we also compare \compas~with well-known node/edge preserving sampling methods available for streaming graphs to show that these do not automatically preserve the community structure thus necessitating the design of \compas~. 

\item We  do a detailed micro-analysis to comprehend the reasons behind superior performance of \compas.
We also show additional benefits of \compas~through an application -- 
%(i) selection of right community detection algorithm for large graphs, and
%(ii) message spreading in streaming graph, 
 selection of (limited) training set for online learning. We obtain a performance that is within \iffalse 95.6\% and\fi 90.5\% of the most informed algorithm GA available for static graphs. %\tb{
 %For reproducibility, we have shared the resources (including code) related to our work in an anonymous link~\cite{si}.

%\item \tb{For reproducibility, we will publicly share the resources (inlcuding code) once the paper gets accepted}   
 \iffalse for first and second applications respectively.\fi
%\TODO{Are the benefits compared to some baselines? Then again report improvements here.}
\end{itemize}









