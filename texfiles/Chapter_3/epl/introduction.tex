\noindent
\section{Introduction}
Information diffusion is one of the most common phenomena that occurs on a
network and the most elementary model of this process is SI
(Susceptible-Infected)~\cite{anderson1992infectious} and its different variations~\cite{watts2002simple,dodds2004universal,pnas1,pnas2,tang2009epidemic,son2012percolation}.
Epidemic/adoption models have recently found a renewed interest with the formulation of temporal networks and the 
realization that most of the real-world networks are temporal in nature (network structure changes with time). 
\iffalse
For example,  Takaguchi et. al. in~\cite{takaguchi2013bursty} presents a model whereby adoption behavior 
of a node is driven by the number of recent contacts with already adopted individual. 
Through simulations on real-world temporal networks the authors further show that burstiness~\cite{karsai2011small} 
affects spreading rate. Similar empirical study was also performed by Karimi et. al. in~\cite{karimi2013threshold}. A further modification to the 
model has been proposed by Backlund et. al. in~\cite{backlund2014effects} which considers that adoption is driven by the number of contacts
with different adopted neighbors within a chosen time instead of a particular neighbor multiple times. Further, information diffusion on temporal 
networks, more specifically prevalence, have also been studied in \cite{rocha2013bursts}. 
Several other results on the study of 
epidemics in temporal networks are listed in~\cite{masuda2013predicting} and on epidemic thresholds in~\cite{prl1,van2012epidemic,zhang2014susceptible}.
This type of history-dependent thresholding spreading pattern is largely observed in spread of bacterial diseases such as tuberculosis and 
dysentery \cite{joh2009dynamics} and 
also in peer-to-peer \cite{sanghavi2007gossiping} and Bittorrent systems \cite{qiu2004modeling}. In \cite{romero2011differences} the authors show that people 
accept ideas/news after repeated exposure.
\fi

The fundamental difference between static and temporal epidemic (SI) models is that in temporal models every agent  within a population is {\em not equally susceptible} to a disease or equally amenable to a rumor
 - the one which has been exposed more number of times (in recent past) are more amenable. This difference however is not well formulated and hence not well modeled - the primary contribution of this work 
is to succinctly define the problem in  terms of a simple model and then theoretically calculate the rate of spread of the epidemics. 
%Albeit a deep literature exists on the study of epidemics in temporal networks, they are mostly simulation based and to the best of our knowledge, a theoretical foundation on the topic is still wanting. 
We consider a spreading model in the lines of~\cite{takaguchi2013bursty} where each susceptible node needs to communicate with the 
infected nodes {\em multiple times} to contract infection. More importantly, unlike memoryless systems, we assume that each node comprises a memory which 
keeps track of the number of contacts it makes with the infected ones. Note that in our system memory is a property which allows each node to remember the number 
of contacts it has already made with the infected ones. While Karimi et. al. studied the effect of burstiness in the diffusion process, we 
are more interested in estimating both empirically and analytically the diffusion time and rate.
Our model is also completely different from  threshold models~\cite{granovetter1978threshold,sur1} where a node
 gets infected when majority of its neighbors are infected or probabilistic SI-models where an infected node,  on coming in contact, 
infects a susceptible node with a probability $p$,  as those are memoryless systems and the transition depends only on the activity of the present time step. 
Our diffusion model also differs from the Neighborhood Exchange (NE) model proposed in \cite{volz2007susceptible}. NE assumes at any given time, an individual
will be in contact with an individual-specific number of neighbors with whom disease transmission is possible while our model considers that a node can be in contact with 
only a single individual. Also NE does not consider the fact that multiple contacts are required to contract infection.

The analytical results are obtained 
considering simple yet important topologies like the complete graph and the infinite $d$-regular trees which accounts for two extreme variants of network topology in terms of edge 
density. 
We demonstrate that the total diffusion time required for complete graph topology (with $n$ nodes) scales as $n^{\frac{k-1}{k}}$ where $k>1$ is the number 
of contacts required to infect a susceptible individual. 
This is in sharp contrast to the case with $k$ = $1$, where it has been shown that diffusion time 
scales as $\log{(n)}$ ~\cite{rumarSreadingPushPull,rumourSpreading_evolvingGraph_PushPull}.
Another important inference we draw from the theoretical analysis is that irrespective of the topology the diffusion process could be divided into two phases: (i) an initial phase where the diffusion rate is very slow and 
(ii) a residual phase where the diffusion becomes very fast.
This inference is also a crucial contribution of this work which can help in containing 
the spread of infectious diseases with minimum overhead if acted upon in the initial phase which we further prove through a detailed empirical study.


\medskip
