\noindent

\section{Innoculation strategy}
%\label{discussion}

% \begin{figure}[htpb]
%   \centering
%   \includegraphics[scale=0.28]{figs1/plot_var_k.eps}
%   %\includegraphics*[scale=0.28]{figs1/plot_var_k.eps}
%  %\includegraphics*[scale=0.15]{figs/T1_vs_exp_T1_d_reg.eps}
%  
% %\hspace{5mm}(a)\hspace{75mm}(b) 
%  \caption{\label{fig_k} Number of nodes at each stage of infection versus time for complete graph of $1000$ nodes with $k=4$. Although the creation of infected nodes is slow initially, the number of 
%  partially infected nodes ($0<p<4$) increases rapidly.}
% \end{figure}


We have defined the epidemic setting aptly fitting the temporal network system whereby agents remember
interactions from the previous time-steps and only adopt an idea after encountering it multiple times. 
%hence observe that for a system consisting of agents who require persuasion (through multiple contacts) before they adopt an idea, 
We find that such information diffusion process undergoes two phases with a slow initial phase followed by a very fast residual phase. This behavior is observed 
irrespective of the underlying topology (refer to figure~\ref{fig1}, simulations done for scale-free graphs~\cite{barabasi1999emergence} and ER-random graphs).
 The reason behind such behavior is that during the process when the first few nodes get fully infected a large fraction of nodes also simulataneously get partly infected as 
represented in figure \ref{tree_ratio}(b). Once the first set of infected nodes are created these partially infected nodes also get quickly infected and this 
results in a sudden ramp-up in the diffusion rate.

The above observations indicate 
that for such systems spreading could be controlled/contained while the system is still in the slow initial phase. 
Accordingly, we perform an empirical study, whereby, we reduce the level of infection of the agents at one particular time step (say $t$) and then estimate the time required to obtain the first sender vis-a-vis the time $t_1$ when 
such action is not initiated. 
Reduction of infection means removing one packet from the chosen (say with probability $\bar p$)  agent, for example, if an agent has acquired $j$ packets at time $t$, we reduce it to $j$ - 1.  
%(reduce the number of packets by 1) with some predefined probability at a given time during the diffusion process and then estimate $t_1$. 
In specific we considered a complete graph with $1000$ nodes and at $t$ =  $0.5 \ast t_1$  
we reduce the level of infection in each node with a probability ($\bar p$) and measure the corresponding time (say  $t_{1}^{\bar p}$). 
In figure \ref{tree_ratio}(a) we plot $\frac{t_{1}^{\bar p} - t_1}{t_1}$ for different values of $\bar p$. 
We observe that creation of the first sender (other than initiator) i.e.,
beginning of the residual phase (where the rate of spread is increased manifold) could be delayed by almost $20\%$ with a probability of $0.3$. 
We show the experiment by inoculating at one particular time step, however a continuous low-grade innoculation strategy can be
initiated and we believe a threshold can be derived whereby the first sender creation can be pushed to infinity. However that 
can be an interesting research direction to be pursued in the future.

\if{0}
\revision{To further verify our hypothesis we 
performed an empirical study, whereby, we randomly removed a certain fraction of infected nodes and at certain time points and then checked what fraction of the whole population 
got infected after a given period of time. Note that here removal of infected node is same as immunizing them. 
In specific we considered a complete graph with $1000$ nodes and several cases separately in each of which we removed a certain 
fraction of infected nodes ($0.5\%$ of the population ($n$)) at different time instances ($t_{1}, 1.2\ast t_{1}, 1.4 \ast t_{1}\ldots  $) and calculated the total number of 
infected nodes at time $3\ast t_{1}$ (refer to figure \ref{tree_ratio}(a)). At $t_1$ there are only two infected nodes in the system, on removing them the infection does not spread 
at all. In the second case when the infected nodes are removed at $1.2\ast t_1$ the infection spread only to a very small fraction of nodes till $3\ast t_{1}$. 
But removal of infected nodes beyond this point ($1.4\ast T_{1}$) only marginally affected the infection spread. These results indicate that the 
spread of infection could be contained if acted upon during the slow initial phase.}


We believe that our findings could open up paths to a number of future studies especially regulating contagion processes in systems with memory. 
We note that our analysis does not 
consider the fact that the extent of influence decays with time as observed in several real-world diffusion processes. We believe such 
investigation calls for additional research efforts.
\fi
% In subsequent works we plan to provide a detailed analytical estimate for
% the ER-random graphs and the BA-scalefree graphs. In its current form our
% model does not take into consideration the fact that the extent of influence
% decays with time as shown in ~\cite{takaguchi2013bursty}. We plan to
% incorporate this into our model and check how it influences the diffusion
% time given an underlying topology. We would also like to extend our analysis
% to Susceptible-Infected-Recovered (SIR) epidemic spreading models as well.

\medskip
