\noindent

Efficient data dissemination in dynamic and distributed networks like delay-tolerant, peer-to-peer, ad-hoc networks is useful for many applications and the 
existing studies have shown combining the push and the pull epidemic protocols can be 
an efficient approach because of their inherent robustness. 
In an interesting study in~\cite{broadcastCoverage_push_pull}, it has been
shown that the cost (in terms of time and resource utilization) to reach the
last 10\% of the agents is much higher compared to the cost of reaching the
first 90\% of the agents during broadcasting in DTNs. To solve this problem,
the authors in this study propose a controlled broadcasting technique where
the broadcast start using push technique, and towards the end of broadcast
the agents switch to adopt the pull technique (instead of push). 
The study in~\cite{push_pull_contentDistribution} has proposed a cooperative file sharing
mechanism in a practical DTN framework, where only a few agents have
Internet connectivity. 
The authors show that, to efficiently spread a file in
the network, the agents that are actually downloading the files from
Internet or cellular networks can use the pull technique, and then they
later on can push the received message to other agents. 
The study in~\cite{DTN_framework_pushPull} has proposed a framework for DTNs, where
the agents in the network form communities located in different geographical
regions, and the agents in different communities use different network
technology such as Bluetooth, WiMAX, etc. Agents located within the same
region may use the pull technique for information sharing, whereas to spread
it across the different regions they can use the push technique.

In addition, there have also been few works on fragmentation of larger sized messages to make them 
suitable for transmission in DTNs and Peer-to-peer networks. In  
~\cite{mikko} the authors introduce various strategies of message fragmentation independent of routing algorithm and evaluate their impact in DTNs. 
Some of the more recent works include ~\cite{altamimi2014message,ginzboorg2014message}. 
%But they mostly deal with on-time delivery of message rather than efficient message dissemination to the whole network.
In peer-to-peer systems there are also numerous instances where the authors 
have tried to efficiently combine the push and the pull epidemic protocols. In ~\cite{sanghavi2007gossiping} the authors propose a combined strategy 
by interleaving between push and pull and show that $k$ pieces can be disseminated from a single source to $n$ users in $9(k+\log n)$ time. 
\if{0}
In this strategy the interleaving between push and pull is achieved by the users trying to push during odd time slot and pull during the 
even time slots with some added constraints. The strategy does not take into consideration the number of useless contacts while data dissemination.
\fi
~\cite{felber2012pulp} introduces a system called Pulp which proposes a data dissemination approach 
by limiting push and also reducing the redundant pulls. 
Some other data dissemination strategies combining push and pull have been proposed in 
~\cite{lo2008some,ozkasap2009principles}. 
Since peer-to-peer systems deal with dissemination of large files fragmenting them before spreading is an obvious choice as is considered in all the 
above works.
%\vspace{-1.5mm}

In this paper, we study for the first time, the systematic
coupling of two different issues that have been dealt in the literature
only independently and in parts. These two issues concern
(a) the effect of message segmentation and (b) the technique adopted for transferring
the message, especially the give-up technique allowing for a
deterministic termination in a completely distributed fashion on broadcast time and wastage.

\medskip
