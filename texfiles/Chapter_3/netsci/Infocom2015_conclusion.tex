\noindent
Our contributions in this chapter can be summarized as - 
\begin{itemize}
 \item We proposed a threshold based diffusion model whereby a susceptible individual contracts infection only after multiple contacts with infected individuals motivated 
 from the idea that an individual adopts an idea after encountering it multiple times.
 \item Irrespective of the topology, the diffusion seems to progress in two phases - (i) initial phase where the diffusion rate is slow followed by (ii) residual phase where 
 the diffusion rate increases manifold.
 \item Through detailed empirical evaluations we showed that for these systems spreading could be controlled if acted upon during the initial phase. In fact  
 we believe that our findings could open up paths to a number of future studies towards regulating contagion processes in systems with memory
 \item Based on our diffusion model we could further propose a set of message broadcast algorithms in dynamic networks. In fact we could come up with a practical strategy which 
can optimize the dual (conflicting) objective of speed and wastage. 
 \item We explored the impact of the problem in different topologies and surprisingly noticed that mere dense 
topology is not of much help in a dynamic setting.
\end{itemize}


\if{0}
The significance of the paper lies in  defining a new problem in the space of information dissemination and broadcast in unstructured dynamic networks. 
In this paper we show through simulation and initial analytical results that
 the speed at which segmented data can be disseminated 
over dynamic network is different from it comprising a single segment. 
We explore the impact of the problem in different topologies and surprisingly notice that mere dense 
topology is not of much help in a dynamic setting. We also try to propose a practical strategy which 
can optimize the dual (conflicting) objective of speed and wastage. 
We believe these initial results can be enriched by  tackling the problem in a more practical setting 
like allowing more than one message transfer in one time step, considering the order of the 
message segments, considering wider variants of topology snapshot etc. 
This along with more rigorous theoretical analysis would be our immediate future work. 

We believe that our findings could open up paths to a number of future studies especially regulating contagion processes in systems with memory. 
We note that our analysis does not 
consider the fact that the extent of influence decays with time as observed in several real-world diffusion processes. We believe such 
investigation calls for additional research efforts.




In this paper we presented a systematic and rigorous study of two crucial issues in dynamic networks which have 
so far been treated independently and in parts. Here, we propose for the first time the effects of coupling these 
two issues -- the effect of message size and the partition structure of the message and the technique adopted for message transfer, i.e., B-P, X-P-P and 
P-P-G. One can also develop several variants of these algorithms and we plan to investigate these in forthcoming works.
We introduced the concept of give up which allows the system that is both dynamic and distributed 
to reduce wastage and deterministically terminate broadcast using some local information at each node. 
We simulate the algorithms on Gnutella snapshots and compare them based on broadcast delay and wastage.
 We show that for a complete graph topology and a suitable partition structure, 
 the broadcast time scales as $n^{\frac{k-1}{k}}$ for the B-P technique. 
 %We make explicit calculations for 
% the minimum non-trivial choice of $m=k=2$ and indicate pointers to obtain the general case.
Further, we study the effect of  
network topologies, e.g., $d$-regular tree, $d$-regular graph, random graph $G(n, p)$ on the evaluation metrics and land up to a 
remarkable observation that one can find a critical value of $d$ for which the dynamics becomes fast.
Even when the topology of the network is not known in advance, we observe that it is better to have lesser number of links 
between the nodes as sparser variants provide a faster dynamics than the denser ones.

\vspace{-1mm}
The current work unfolds many new directions of research and we plan to report a majority of these in a forthcoming paper. 
% A first step would be to conduct a detailed theoretical analysis of the broadcast time for the general 
% choice of $m$ and $k$. 
We plan to investigate the broadcast delay and wastage analytically which has been established here only 
through numerical evidence. All our results are based on the assumption that no data loss occurs during transmission. However, in 
any wireless ad-hoc network data loss is a common phenomena. We also assumed that only a single packet is transferred in each 
contact opportunity although for dynamic networks like DTN it may be more than one. We plan to address these issues in our subsequent works as well.
Further, we plan to investigate the importance of the segment order, i.e., an agent is allowed to 
become a sender only when it receives the segments in a particular order. 
\fi

\medskip
