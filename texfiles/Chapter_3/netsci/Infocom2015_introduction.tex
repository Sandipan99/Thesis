\noindent
{\bf Broadcast algorithm based on diffusion model:} The study of broadcast over unstructured and mobile networks always assumes that the 
size of the message is  small enough to be transfered from one node to other 
on the short durations of contacts between the nodes. 
Contrary to this, we here explore
 the idea of ``segmented messages'' where we assume that the duration of a contact 
between the nodes is not always sufficient for the transfer and therefore the message might need to be 
segmented/divided into sub-parts and sent individually. 
At the ethernet level such techniques of segmented broadcast are often termed as pipelined broadcast 
~\cite{patarasuk2008techniques, watts1995pipelined}. Note that this is equivalent to diffusion model we proposed earlier with number of contacts ($k$) replaced by number 
of packets in a message. 
We systematically study the effect of the size and partition structure of the message on the broadcast time.
 %considering different variants of the basic push-pull transfer protocol. 
% \textcolor{red}{Note that we essentially deal with a theoretical abstraction of the problem of segmented message spreading and show that it is 
% significantly different from the classical epidemic spreading.}{\bf we have not shown this.} \todo{Have we shown this?}
%The transmission algorithms that we propose might require suitable adjustments when it is adapted for a specific type of a dynamic network.      

%\todo{The two paragraphs need to be discussed, n, m, k, s are extremely confusing and need to be simplified}
{In specific, we investigate in details the effect of message segmentation as well as the message transfer protocols on the overall 
broadcast delay and message wastage. 
We assume that a big file is split into $k$ ($>1$) packets and 
%the packets are grouped into $s$ ($\geq 1$) segments, each containing $k$ packets. 
at the beginning, there is only one sender node in the network that has all the packets. 
Further, a node can transfer only one  packet in a single contact opportunity, and it can do so only when it has received all the 
packets constituting at least one segment of the message. When all the nodes present in the network have eventually received all the packets, 
broadcasting is assumed to be complete. 

Initially, we investigate the push transfer protocol whereby messages are `pushed'
by the node holding a message to the node not having the message ~\cite{demers1987epidemic,lo2008some}. 
We attempt to study the effect in different types of topologies e.g., complete graph, $d$-regular tree, $d$-regular graph, random graph (with average degree $d$). 
%For a complete graph topology of size $n$ and the simple push case  
%we show that the overall time required for the broadcast in case of segmented messages scales as $n^{\frac{k-1}{k}}$ where $k>1$ is the number of packets. 
%This is in sharp contrast to the single message case ($k$ = $1$) where it has been shown that broadcasting time 
%scales as $\log{(n)}$ ~\cite{rumarSreadingPushPull,rumourSpreading_evolvingGraph_PushPull}. 
A remarkable observation is that in topologies like $d$ regular tree, $d$ regular graph and random graph,
for even the simple push transfer protocol, one can find an optimal value of $d (d>1)$, for which the broadcast delay and wastage is minimum. (section ~\ref{suggestions}).
As a corollary, through simulations on real traces, we identify that for two networks with the same number of nodes, broadcast 
time required is far smaller for the one with lesser edge density. 
%On removing some edges from the original network while maintaining the network connectivity 
%substantially reduces the broadcast time.
This finding, 
we believe, indicates a very crucial point -- a sparse communication network per se is not disadvantageous. 

However, we observe that push transfer protocol results in a large number of useless contacts.  
%In order to reduce the same, we propose in this paper, a ``give-up'' mechanism whereby nodes attempt to discover 
%their neighborhood by maintaining a history of all previous contacts and stop attempting to perform any further 
%push after a certain number of unique unsuccessful contacts (reduces to 10\%, message size = 2). 
An unsuccessful/useless contact here refers to a case 
where a sender node attempts to send a packet to another node who already has the packet. 
In order to reduce both broadcast time and wastage we propose a combined strategy whereby the nodes in the system initially push and then switch over 
to pull after a certain percentage (say $x$) of the nodes have received the full message. We observe that if $x$ is carefully chosen, both gain in  
broadcast time and reduction in wastage is achieved. However, to determine that $x$\% of the nodes have indeed received the message, the system needs to maintain a global information 
which is not feasible in a distributed setting like this.
% An unsuccessful/useless contact here refers to a case 
% where a sender node attempts to send a packet to another node who already has the packet. 
% Nevertheless, such a ``give-up'' strategy results in drastic reduction in the number of nodes 
% finally receiving the message especially for larger values of $m$ (55\%, message size = 16). 
In order to circumvent this problem, we introduce a distributed version of the previous technique along with ``give-up'' mechanism 
whereby nodes attempt to discover 
their neighborhood by maintaining a history of all previous contacts and stops participating in the broadcast 
after a certain number of unique unsuccessful contacts.  
This algorithm when tested over Gnutella topologies is found to yield  lesser broadcast delay 
without significant increase in wastage.
We believe, our algorithm with minor modifications is applicable to a 
wide range of dynamic networks ~\cite{eugster2003many,jin2010epidemic,sanghavi2007gossiping}.



\medskip
